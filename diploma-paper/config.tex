%!TEX root = graduate-work.tex
% Добавьте ссылку на файлы с текстом работы
% Можно использовать команды:
%   \input или \include
% Пример:
%    \input{mainfiles/1-section} или \include{mainfiles/2-section}
% Команда \input позволяет включить текст файла без дополнительной обработки
% Команда \include при включении файла добавляет до него и после него команду
% перехода на новую страницу. Кроме того, она позволяет компилировать каждый файл
% в отдельности, что ускоряет сборку проекта.
% ВАЖНО: команда \include не поддерживает включение файлов, в которых уже содержится команда \include,
% т.е. не возможен рекурсивный вызов \include
\newcommand*{\Source}{
    %!TEX root = ../graduate-work.tex
\phantomsection
\section*{Введение} 
\addcontentsline{toc}{section}{Введение}
Внедрение глубокого обучения в критически важные для безопасности системы (safety-
critical systems) — от авионики и автономного вождения до медицинской диагностики и
управления энергосетями — создало фундаментальный парадокс. С одной стороны,
нейронные сети демонстрируют производительность, превосходящую человеческую, с
другой — они остаются "черными ящиками", подверженными катастрофическим сбоям
при незначительных возмущениях входных данных, известных как состязательные
примеры (adversarial examples)\cite{Guidotti2019Enhancing}. В отличие от традиционного программного
обеспечения, где логика задается явно, поведение нейронной сети определяется
миллионами весовых коэффициентов, что делает невозможным использование
классических методов тестирования для гарантии отсутствия ошибок. Формальная
верификация призвана решить эту проблему, предоставляя математическое
доказательство того, что сеть удовлетворяет заданным спецификациям для бесконечного
множества возможных входных данных.

\subsection{Формальная формулировка задачи верификации}

В наиболее общем виде задачу верификации нейронной сети можно определить как проблему 
доказательства выполнимости логического следования. 
Рассматривая нейронную сеть как функцию 
$f: \mathcal{D}_{in} \to \mathcal{D}_{out}$, где $\mathcal{D}_{in} \subseteq \mathbb{R}^n$ и $\mathcal{D}_{out} \subseteq \mathbb{R}^m$, 
мы стремимся проверить, что для любого входного вектора $\mathbf{x}$, 
удовлетворяющего некоторому предусловию $\phi(\mathbf{x})$, 
выходной вектор $\mathbf{y} = f(\mathbf{x})$ удовлетворяет постусловию $\psi(\mathbf{y})$.
Формально задача верификации заключается в проверке истинности утверждения:

$$\forall \mathbf{x} \in \mathcal{D}_{in}, \quad \phi(\mathbf{x}) \implies \psi(f(\mathbf{x}))$$

На практике верификаторы часто решают обратную задачу — задачу поиска контрпримера (falsification). 
Они пытаются найти такой вектор $\mathbf{x}$, для которого выполняется предусловие $\phi(\mathbf{x})$, 
но нарушается постусловие $\psi(f(\mathbf{x}))$. То есть, они проверяют выполнимость формулы (Satisfiability):

$$\exists \mathbf{x} \in \mathcal{D}_{in} : \phi(\mathbf{x}) \land \neg \psi(f(\mathbf{x}))$$

Если такая формула невыполнима (UNSAT), то свойство верифицировано. 
Если выполнима (SAT), то найденный $\mathbf{x}$ является состязательным примером или доказательством небезопасности системы.\cite{marzari2025advancingneuralnetworkverification}

\textbf{Спецификации локальной устойчивости и безопасности}

Наиболее распространенным классом свойств, верифицируемых в современных исследованиях, 
является локальная робастность (local robustness). Она подразумевает, что малые 
возмущения входного сигнала не должны изменять решение сети. 
Для задачи классификации с входным образом $\mathbf{x}_0$, классифицируемым как класс $c$, 
свойство робастности в $\epsilon$-окрестности по норме $L_p$ формулируется так:

\begin{itemize}
    \item \textbf{Предусловие} $\phi(\mathbf{x})$: $||\mathbf{x} - \mathbf{x}_0||_p \le \epsilon$. Это определяет гиперкуб (для $L_\infty$) или гиперсферу (для $L_2$) возможных входных данных.
    \item \textbf{Постусловие} $\psi(\mathbf{y})$: $\forall j \neq c, f(\mathbf{x})_c > f(\mathbf{x})_j$. Это означает, что оценка (логит) правильного класса $c$ должна оставаться строго больше оценок всех остальных классов. \cite{DBLP:journals/corr/abs-2002-03339}
\end{itemize}

Помимо робастности, существуют более сложные функциональные спецификации безопасности, часто применяемые в управлении. 
Примером служит система предотвращения столкновений ACAS Xu, где свойства формулируются в терминах физических переменных.\cite{Demarchi2023SupportingStandardization}

\subsection{Стандарт VNN-LIB: Унификация языка спецификаций}

До 2020 года развитие области сдерживалось фрагментацией: каждый
исследовательский инструмент использовал свой формат для описания нейронных сетей
и проверяемых свойств, что делало невозможным прямое сравнение
производительности. В ответ на это сообщество разработало стандарт VNN-LIB
(Verification of Neural Networks Library), который стал де-факто международным
стандартом для описания задач верификации.\cite{VNNLIBWebsite}
VNN-LIB базируется на философии SMT-LIB (стандарта для SMT-решателей) и
использует S-выражения для декларативного описания ограничений. Стандарт
определяет три ключевых компонента:
\begin{itemize}
    \item \textbf{Синтаксис запросов:} Позволяет определять входные и выходные переменные, их типы (Real, Int) и накладывать на них линейные и нелинейные ограничения.
    \item \textbf{Семантика:} Строго фиксирует интерпретацию логических связок (and, or, not, =>) и арифметических операций, устраняя двусмысленность при обработке различными решателями.
    \item \textbf{Интерфейс решателей:} Стандартизирует протокол взаимодействия между инструментами и тестовыми средами, что критически важно для автоматизации соревнований VNN-COMP.
\end{itemize}
Пример спецификации на языке VNN-LIB, описывающий ограничение входного значения X некоторым диапазоном и проверку того, что выход Y не превышает пороговое значение:
\begin{lstlisting}[caption={Пример SMT-LIB},label={lst:smtlib}]
    (declare-const X Real)
    (declare-const Y Real)
    (declare-const Y_out Real)
    ; Предусловие: вход в диапазоне
    (assert (>= X 0.0))
    (assert (<= X 1.0))
    ; Связь переменных (абстрактная)
    (assert (= Y_out (network_output X)))
    ; Постусловие (проверка на нарушение): выход > 5
    (assert (> Y_out 5.0))
\end{lstlisting}
Использование VNN-LIB в сочетании с форматом ONNX для обмена моделями нейронных
сетей позволило создать экосистему, в которой бенчмарки и инструменты могут
разрабатываться независимо друг от друга.\cite{arXiv2212VNNCOMP2022}

\subsection{Вычислительная сложность и теоретические ограничения}

Анализ вычислительной сложности задачи верификации раскрывает фундаментальные трудности, с которыми сталкиваются алгоритмы. 
Сложность напрямую зависит от типа функций активации, используемых в сети.
NP-полнота для ReLU-сетей:
Для сетей, использующих кусочно-линейные функции активации, такие как Rectified Linear Unit (ReLU: $\max(0, x)$), 
задача верификации доказана как NP-полная.\cite{arXiv2403RobustnessVerification}
\begin{itemize}
    \item \textbf{Доказательство:} Задача выполнимости булевых формул (3-SAT) может быть полиномиально сведена к задаче верификации ReLU-сети. Нейрон ReLU способен эмулировать логический вентиль, а слои сети — логическую схему. Поэтому в худшем случае время верификации экспоненциально возрастает относительно числа нейронов.
    \item \textbf{Последствия:} Это означает, что не существует полиномиального алгоритма, гарантирующего верификацию произвольной ReLU-сети (при условии, что $P \neq NP$). Однако аналогично SAT-решателям, многие практические инстансы задач обладают структурой, позволяющей решать их существенно быстрее худшего случая.
\end{itemize}

$\exists\mathbb{R}$-полнота и неразрешимость для нелинейных сетей:
Для сетей с гладкими нелинейными активациями (Sigmoid, Tanh, GELU) ситуация еще сложнее.
\begin{itemize}
    \item Задача верификации таких сетей относится к классу сложности $\exists\mathbb{R}$-complete (Existential Theory of the Reals).\cite{NeurIPS2021ExistsRComplete} Этот класс включает задачи, сводимые к поиску решений систем полиномиальных неравенств над вещественными числами. Он находится между NP и PSPACE.
    \item Более того, для некоторых классов рекуррентных сетей (RNN) или сетей с трансцендентными функциями задача достижимости может быть алгоритмически неразрешимой в общем виде.\cite{IJCAI2018ReachabilityDNN} Это накладывает теоретический предел на возможности полной верификации таких архитектур и вынуждает использовать методы аппроксимации (абстрактную интерпретацию), которые жертвуют полнотой ради разрешимости.
\end{itemize}
Таким образом, задача верификации представляет собой поиск баланса между выразительностью модели и вычислительной стоимостью доказательства её свойств.


    \section{Анализ алгоритмов и верификаторов}

\subsection{Alpha-Beta-CROWN}

Alpha-Beta-CROWN представляет собой вершину эволюции методов распространения границ (Bound Propagation). 
Его успех базируется на двух ключевых компонентах, обозначенных греческими буквами в названии. \\
$\alpha$-CROWN (Alpha): Оптимизация наклонов \\
Базовый алгоритм CROWN (эквивалентный DeepPoly) использует линейные релаксации для функций активации. 
Для ReLU $y = \max(0, x)$ при $x \in [l, u]$ (где $l < 0 < u$) нижняя граница описывается 
прямой линией $\lambda x$, где $0 \le \lambda \le 1$. Т
радиционные методы фиксируют $\lambda$ эвристически (например, $\lambda = u/(u-l)$). \\
$\alpha$-CROWN трактует $\lambda$ как оптимизируемый параметр $\alpha$. 
Алгоритм запускает градиентный спуск (аналогичный обучению сети) для поиска таких значений 
$\alpha$ для каждого нейрона, которые максимизируют нижнюю границу выхода сети (или минимизируют верхнюю). 
Это позволяет получать максимально точные (tight) границы без ветвления, 
что часто достаточно для верификации "легких" свойств. \cite{AlphaBetaCrownVNNCOMP23}
$\beta$-CROWN (Beta): Полная верификация без LP \\
Когда $\alpha$-CROWN недостаточно, требуется ветвление (Branch-and-Bound). Традиционные методы (как MIPVerify) 
решают в каждом листе дерева поиска задачу линейного программирования (LP), что медленно и требует переноса данных с GPU на CPU.
$\beta$-CROWN решает эту проблему, кодируя ограничения ветвления ("нейрон $i$ > 0" или "нейрон $i$ <= 0") 
непосредственно в процедуру распространения границ через двойственные переменные Лагранжа ($\beta$). Это позволяет:
\begin{itemize}
    \item Избежать вызова внешних LP-решателей.
    \item Выполнять оценку границ для тысяч веток параллельно на GPU как серию матричных умножений.
    \item Оптимизировать параметры ветвления $\beta$ совместно с $\alpha$ через градиентный спуск. \cite{AlphaBetaCrownVNNCOMP23}
\end{itemize}

\subsection{VeriNet и nnenum}
VeriNet фокусируется на методе Symbolic Interval Propagation (SIP). 
В отличие от CROWN, который распространяет линейные неравенства, SIP сохраняет зависимости в виде символьных выражений. 
VeriNet особенно эффективен в стратегии Input Splitting (расщепление входного пространства), 
что делает его предпочтительным выбором для задач с малой размерностью входа (робототехника, контроллеры ACAS Xu), 
где можно покрыть входное пространство сеткой. 
Алгоритм также использует итеративное уточнение границ, минимизируя накопление ошибки овер-аппроксимации.\cite{IntervalReachabilityAnalysis} \\

nnenum (победитель VNN-COMP 2020) использует стратегию Abstraction Refinement. 
Инструмент динамически переключается между различными представлениями множеств:
\begin{itemize}
    \item Zonotopes: Быстрые, но грубые.
    \item Star Sets (ImageStars): Точные, но вычислительно дорогие.
    \item Polyhedra: Используются при необходимости максимальной точности.
\end{itemize}
nnenum был первым инструментом, который показал, что умное управление уровнем абстракции может быть эффективнее, чем использование одного (пусть и мощного) метода для всей сети.\cite{NNEnumPaper}


\subsection{Marabou 2.0 и NeuralSAT}

Marabou 2.0 представляет собой современное развитие SMT-подхода. 
Ключевым нововведением стала замена классического поиска на процедуру DeepSoI (Sum of Infeasibilities). 
Вместо бинарного поиска выполняющего присваивания, DeepSoI минимизирует непрерывную функцию, 
представляющую "суммарное нарушение" ограничений ReLU. Это позволяет алгоритму "скользить" к решению, 
используя градиентную информацию, что значительно быстрее чистого комбинаторного перебора.\cite{arXiv2401Marabou2} \\

NeuralSAT переносит в нейросетевую верификацию мощь современных SAT-решателей, использующих CDCL (Conflict-Driven Clause Learning). 
В процессе поиска NeuralSAT выявляет несовместные комбинации состояний нейронов 
(например, "если нейрон A активен, то нейрон B не может быть активен") и добавляет их в базу знаний 
как "конфликтные клаузы" (lemmas). Это предотвращает повторный просмотр заведомо тупиковых ветвей поиска в будущем, 
обеспечивая прунинг (отсечение) огромных поддеревьев. 
Этот подход особенно эффективен для сетей со сложной внутренней логической структурой.\cite{NeuralSATRepo}


\subsection{GPU-ускорение и библиотека \texorpdfstring{\texttt{auto\_LiRPA}}{auto LiRPA}}
Современные достижения в верификации были бы невозможны без глубокой оптимизации алгоритмов под аппаратное обеспечение.
\subsubsection{GPU-ускорение и библиотека \texorpdfstring{\texttt{auto\_LiRPA}}{auto LiRPA}}
Традиционные алгоритмы верификации (Simplex, SMT) по своей природе последовательны и требуют сложного 
управления потоком управления (branching logic), что 
плохо подходит для архитектуры GPU (Single Instruction, Multiple Data - SIMD).
Прорыв произошел с созданием библиотеки \texttt{auto\_LiRPA}.\cite{AlphaBetaCrownVNNCOMP23}

\begin{itemize}
    \item \textbf{Идея:} Представить алгоритмы распространения границ (Bound Propagation) как операции над тензорами в вычислительном графе (аналогично тому, как PyTorch выполняет forward/backward pass).
    \item \textbf{Реализация:} Библиотека автоматически строит граф вычислений для верхней и нижней границ любого узла сети. Это позволяет верифицировать любую архитектуру, поддерживаемую PyTorch (CNN, ResNet, Transformer, LSTM), без написания специализированного кода.
    \item \textbf{Batch BaB:} В методе Branch-and-Bound вместо последовательной обработки веток верификатор собирает пакет (batch) из сотен тысяч нерешённых подзадач. Границы для всех этих подзадач вычисляются одновременно за один проход GPU, что амортизирует накладные расходы на запуск ядер CUDA и обеспечивает утилизацию GPU, близкую к 100\%.\cite{LiRPAVerifyRepo}
\end{itemize}

\subsubsection{Метод секущих плоскостей (Cutting Planes)}
Для усиления линейных релаксаций без полного ветвления применяются методы генерации отсечений (Cutting Planes), заимствованные из целочисленного программирования.
\begin{itemize}
    \item \textbf{GCP-CROWN (General Cutting Planes):} Метод добавляет к задаче линейные неравенства, связывающие значения нескольких нейронов (например, входов и выходов слоя). Эти неравенства «отсекают» дробные решения, допустимые в LP-релаксации, но недопустимые для реальной ReLU-сети.
    \item \textbf{BICCOS (Branch-and-bound Inferred Cuts with COnstraint Strengthening):} Специализированная техника для Alpha-Beta-CROWN, которая генерирует отсечения на основе анализа логических импликаций в дереве поиска и применяет их итеративно. Это позволяет решать задачи, которые ранее требовали бы слишком глубокого дерева ветвления, за счёт усиления границ в корневых узлах.\cite{AutoLiRPAPracticeSlides}
\end{itemize}

    \section{Параллелизм в LLM}

\subsection{Задача обучения LLM}

\subsection{Tensor Parallelism (TP)}

\subsection{Pipeline Parallelism (PP)}

\subsection{Fully Sharded Data Parallelism (FSDP)}
    \include{mainfiles/5-parallel-verifiers}
    \section{Дизайн и постановка экспериментов}
    \include{mainfiles/7-conclusion}
}


% Информация о годе выполнения работы
\def\Year{%
    % 2006%
    \the\year%     % Текущий год
}

% Укажите тип работы
% Например:
%     Выпускная квалификационная работа,
%     Магистерская диссертация,
%     Курсовая работа, реферат и т.п.
\def\WorkType{%
    % Выпускная квалификационная работа%
    % Магистерская диссертация%
    % Курсовая работа%
    % Реферат%
    Дипломная работа%
}

% Название работы
%%%%%%%%%%% ВНИМАНИЕ! %%%%%%%%%%%%%%%%
% В МГУ ОНО ДОЛЖНО В ТОЧНОСТИ
% СООТВЕТСТВОВАТЬ ВЫПИСКЕ ИЗ ПРИКАЗА
% УТОЧНИТЕ НАЗВАНИЕ В УЧЕБНОЙ ЧАСТИ
\def\Title{%
    Параллелизмы в верификации нейронных сетей%
}


% Имя автора работы
\def\Author{%
    Воробьев Сергей Юрьевич%
}

% Информация о научном руководителе
%% Фамилия Имя Отчество%
\def\SciAdvisor{%
    Ильюшин Евгений Альбинович%
}
%% В формате: И.~О.~Фамилия%
\def\SciAdvisorShort{%
    Е.~А.~Ильюшин%
}
%% должность научного руководителя
\def\Position{%
    % профессор%
    % доцент%
    % старший преподаватель%
    % преподаватель%
    % ассистент%
    ведущий научный сотрудник%
    % старший научный сотрудник%
    % научный сотрудник%
    % младший научный сотрудник%
}
%% учёная степень научного руководителя
\def\AcademicDegree{%
    % д.ф.-м.н.%
    % д.т.н.%
    к.ф.-м.н.%
    % к.т.н.%
    % без степени%
}

% Информация об организации, в которой выполнена работа
%% Город
\def\Place{%
    Москва%
}
%% Университет
\def\Univer{%
    Московский государственный университет имени М.~В.~Ломоносова%
}
%% Факультет
\def\Faculty{%
    Факультет вычислительной математики и кибернетики%
}
%% Кафедра    
\def\Department{%
    Кафедра информационной безопасности%
}     

%%%% Переключите статус документа для отладки
%%%% В режиме draft документ собирается очень быстро
%%%% и выводится полезная информация о том
%%%% какие строки вылезают за границы документа, что удобно для борьбы с ними
\def\Status{%
    % draft%
    final%
}

%%%% Включает и выключает подпись <<С текстом работы ознакомлен>>
\def\EnableSign{%
    % true%
}
